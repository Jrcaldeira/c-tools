\documentclass[11pt]{article}
\usepackage[a4paper, inner=1.5cm, outer=3cm, top=2cm,
    bottom=3cm, bindingoffset=1cm]{geometry}
\usepackage[utf8]{inputenc}
\usepackage[portuguese]{babel}
\usepackage[T1]{fontenc}
\usepackage{amsmath}
\usepackage{amssymb}

\usepackage{amsfonts} % if you want blackboard bold symbols e.g. for real numbers
\usepackage{graphicx} % if you want to include jpeg or pdf pictures
\usepackage{enumerate} %enumerar
\usepackage{ae}
\usepackage{graphics}
\usepackage{colortbl}
\usepackage{verbatim}
%\usepackage{graphicx}
\usepackage{wrapfig}
\usepackage{caption}
\usepackage{subcaption}
\usepackage{gensymb}
\usepackage{xcolor}
\usepackage{upgreek}
\usepackage{multirow}
\usepackage{amssymb,amsmath}
\usepackage{latexsym}
\usepackage{tabulary}
\usepackage{tabularx}
%tabela toprule precisa
\usepackage{booktabs}
% pra rotacionar
\usepackage{rotate}
\usepackage{rotating}%
\usepackage{array}
\usepackage{parskip}
\usepackage{footnote}
\usepackage{makeidx} %indice remissivo
%\usepackage{showidx}
\usepackage{threeparttable}
\usepackage{marvosym}
\usepackage{multirow}
\usepackage{bigstrut}
\usepackage{textcomp}
\usepackage{cancel}
\usepackage{wasysym}
%\usepackage{marvosym}
\usepackage{tipa}%omega
\usepackage{upgreek}
\usepackage{enumerate}
\usepackage{indentfirst}
\usepackage{hyperref}%coloca referencia marcada
\usepackage{tikz}%fazer graficos
\usepackage{tikz-3dplot}%fazer graficos
\usepackage[right]{lineno}
\begin{document}
\begin{tabular}{m{2.8cm}p{9.4cm}m{4.2cm}}
%\begin{tabular}{m{1.1cm}p{8.7cm}m{2.6cm}}

\centering{\includegraphics[width=1.8cm]{Logomarca_jpg.jpg}}
&\centering{\fontsize{11}{11}\selectfont{\textbf{Universidade Federal de Ouro Preto}}}\\
\centering{\fontsize{9}{9}\selectfont{\textbf{Instituto de Ciências Exatas e Aplicadas}}}\\
\centering{\fontsize{9}{9}\selectfont{\textbf{Departamento de Engenharia Elétrica}}}\\

& \centering{\includegraphics[width=4.0cm]{logoicea.jpg}}
\end{tabular}\\
\hrule

\		

\
Implemente um programa que retorne uma aproximação do valor de $\pi$, de acordo com a Fórmula de Leibniz: 

\begin{displaymath}
\pi \approx 4 . \left(1 - \frac{1}{3} + \frac{1}{5} - \frac{1}{7} + \frac{1}{9} - \frac{1}{11} + \cdots + \frac{1}{n}\right)
\end{displaymath}

Ou seja: 
\begin{displaymath}
\pi \approx 4 . \sum_{i=0}^{n-1} \frac{(-1)^i}{2.i+1}
\end{displaymath}

\

Onde $n$ indica o número de termos da série que devem ser usados para o cálculo do valor de $\pi$. Implemente um programa que receba do teclado o número de termos a ser utilizados no cálculo de $\pi$ e, usando a função acima, calcule e imprima o valor de $\pi$ calculado. Se o número de termos fornecido for menos do que um, o programa deve exibir uma mensagem de erro.

\end{document}